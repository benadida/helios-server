% TEMPLATE for Usenix papers, specifically to meet requirements of
%  USENIX '05
% originally a template for producing IEEE-format articles using LaTeX.
%   written by Matthew Ward, CS Department, Worcester Polytechnic Institute.
% adapted by David Beazley for his excellent SWIG paper in Proceedings,
%   Tcl 96
% turned into a smartass generic template by De Clarke, with thanks to
%   both the above pioneers
% use at your own risk.  Complaints to /dev/null.
% make it two column with no page numbering, default is 10 point

% Munged by Fred Douglis <douglis@research.att.com> 10/97 to separate
% the .sty file from the LaTeX source template, so that people can
% more easily include the .sty file into an existing document.  Also
% changed to more closely follow the style guidelines as represented
% by the Word sample file. 

% Note that since 2010, USENIX does not require endnotes. If you want
% foot of page notes, don't include the endnotes package in the 
% usepackage command, below.

% This version uses the latex2e styles, not the very ancient 2.09 stuff.
\documentclass[letterpaper,10pt]{article}
\usepackage{epsfig,endnotes}

\usepackage[flushleft]{threeparttable}
\usepackage{amssymb}
\usepackage{graphicx}
\usepackage{amsmath}

\usepackage{url}

\newcommand{\keywords}[1]{\par\addvspace\baselineskip
\noindent\keywordname\enspace\ignorespaces#1}

\begin{document}

%don't want date printed
\date{}

% first the title is needed
\title{\Large \bf From Helios to Zeus}

\author{Georgios Tsoukalas \qquad  Kostas Papadimitriou\\ 
Panos Louridas \qquad Panayiotis Tsanakas\\[.5cm]
 Greek Reseach and Education Network (GRNET),\\
56 Mesogeion Avenue, Athens, Greece\\
\{gtsouk,kpap,louridas,tsanakas\}@grnet.gr}


\maketitle

% Use the following at camera-ready time to suppress page numbers.
% Comment it out when you first submit the paper for review.
\thispagestyle{empty}

\subsection*{Abstract}

We present Zeus, a verifiable Internet ballot casting and counting
system based on Helios, in which encrypted votes are posted eponymously
to a server, then are anonymized via cryptographic mixing, and
finally are decrypted using multiple trustee keys. Zeus refines the
original Helios workflow to address a variety of practical issues, such
as usability, parallelization, varying election types, and tallying
through a separate computing system.
In rough numbers, in the first seven months of deployment, Zeus has been
used in about 40 elections across 30 public and private institutions,
tallying a total of about 10000 votes.

\section{Introduction}

Helios~\cite{adida:2008} (\url{http://heliosvoting.org}) is a well
known system for Internet voting, in which the entire process is
carried out through digital means---there is no paper trace, nor
ballots in physical form. Instead, there is a trail of mathematical
proofs linked together that can verify the correctness of every step
in the voting process. Helios has been used in several real world
elections and its basic architectural design has proven robust.

In the summer of 2012 our organisation,{\sc grnet}, was asked to
provide a system for electronic voting to be used in elections in
universities in Greece. Based on its good track record and the body of
work that has been published on it, we decided to see whether Helios
could fit our needs. Unfortunately, we found that it could not,
because of the voting system that would be used in the elections we
were called to run.

We therefore had the problem of creating a system that could be used
for the required elections, to be taken in approximately three months
time. Having no time to scratch from scratch, and having no intention
to re-invent the wheel or to embark on cryptography research, we set
out to create a system based on Helios, as much as possible, with the
necessary modifications to suit our needs. The system came to be known
as Zeus (thus remaining in the pantheon of Greek gods) and has been
used with unequivocal success for many elections over several months
now. That was especially satisfying since the use of Zeus was not
without critics, for reasons that we will explain below.

The contribution of this paper is not any new electronic voting
protocols or an entirely new voting workflow; we have adopted and used
proven systems and ideas that have already been described in the
literature. Our contribution is then:
\begin{itemize}
\item A description of a working e-voting system that has been used in
  many real-world elections, by not especially technically-savvy
  users
\item An account of an e-voting system used in a particularly
  adversarial context.
\item An account of our experience from using the system. Zeus has
  been used in over 40 real-world elections involving more than 10000
  voters in total.
\item The usability and accessibility issues we had to tackle.
\item The internal design and implementation choices.
\end{itemize}

The rest of this paper proceeds as follows. In section~\ref{sec:related},
we discuss related work and the roots of Zeus. In
section~\ref{sec:challenges} we discuss Zeus's background and outline the
practical problems we faced and our approach to solve them. In
section~\ref{sec:workflow} we go through the complete voting workflow
from the users' perspective. In section~\ref{sec:architecture} we
elaborate on the technical details of the voting workflow and its
implementation, from both a cryptographic and an engineering point of
view. In section~\ref{sec:experience} we relate our experiences with
deployment and use of Zeus as a service for elections over the
internet.

\section{Related Work}
\label{sec:related}

Zeus is based on Helios~\cite{adida:2008}, an electronic voting system
developed by Ben Adida. The original Helios publication
\cite{adida:2008} proposed cryptographic mixing via
Sako-Kilian~\cite{sako:1995} / Benaloh~\cite{benaloh:2006}
mixnets~\cite{sako:1995} of ballots for their anonymization. Mixnets
were dropped in version 2 of Helios, which was used for elections at
the Universit\'{e} Catholique de Louvain in 2008, because in the
then-existing implementation they could not accommodate the need for
votes having different weights according to the voter's
category~\cite{adida:2009}. In place of mixnets homomorphic
tallying~\cite{cohen:1985} was adopted, as it was argued that it was
``easier to implement efficiently, and thus easier to
verify''~\cite{adida:2009}.

Homomorphic tallying operates directly on \emph{encrypted} votes,
tallying them all up in a single encrypted result, which is the only
piece of information that is decrypted. This is much easier
computationally but sacrifices generality because the tallying
algorithm must be encoded in the cryptographic processing itself---one
cannot get results in the form of individual ballots as they were
submitted.

On the other hand, mixing shuffles the set of votes in a provably
correct but practically untraceable way, therefore anonymizing it. The
output is the same set of unencrypted ballots as they were before
encryption and submition, only in random, unknowable order. Then any
method of tallying can be applied to compute results. Although Helios
itself is not using mixnets any more, a variant has been developed
that does~\cite{bulens:2011}; however the code is not publicly
available (nor was it available to us in any way).

Over time researchers have explored possible vulnerabilities in
different versions of Helios~\cite{heiderich:2011}. To the best of our
knowledge these attacks are not applicable to Zeus.

More recently, the question of endowing Helios with everlasting
privacy has been explored~\cite{demirel:2012}. Helios's offers
computationally secure privacy, i.e., it is currently prohibitively
expensive, from a computational point of view, to break voter
anonymity. This does not preclude, however, that it will be
impractical in several years' time. Protocols for everlasting privacy
can be used for both homomorphic and mixnet-based elections. Zeus,
like Helios, relies on computational privacy and it would be possible
to explore the adoption of an everlasting privacy approach.

Due to the switch from homomorphic tallying to mixnets and other
requirements and refinements presented below, the Zeus software only
retains 50\% of the original source code, including the entire
homomorphic tallying support, which is unused.

\section{Zeus Background and Challenges}
\label{sec:challenges}

Zeus was initiated after the use of electronic vote was permitted by
decree for the election of the Governing Councils of Higher Education
Institutions in Greece. In each institution the Governing Council is
directly elected by its faculty and is its main governing body. The
election uses the Single Transferable Vote (STV) system, in which
voters do not simply indicate the candidates of their preference, but
also rank them in order of preference. 

When we were charged with providing an implementation of a system
implementing electronic voting we decided to investigate Helios's
suitability, as we needed a mature system with a proven record in real
world elections and published open source code. The current version of
Helios (version 3) allows internet election from end-to-end: from the
moment the voter casts a ballot through a web browser to the
publication of election results. It does that by never actually
decrypting the ballots but performing a series of homomorphic
calculations on them. In the end, the results of the calculations are
decrypted and published. 

Although using a single system for the whole process is appealing, the
use of homomorphic tallying in Helios cannot accommodate voting
systems in which not just the individual choices on the ballot matter,
but the whole ballot itself. In STV, homomorphic tallying could pass
to the STV algorithm the information that a certain candidate has been
selected in rank $r$ by $n$ voters, but this is not enough, as the whole
ballot and not just each rank separately is passed around during STV's
counting rounds.

We realised, however, that it is not necessary to use Helios's
homomorphic tallying capabilities. We decided to use Helios for
\emph{counting the ballots}, not for producing the election results.
Once we do have a veriable ballots count, this can be fed to an STV
calculator, or indeed to a calculator of any voting system. Since the
ballots are published, and the algorithm is also published, a third
party can always verify that the results are correct.

Interestingly, the original publication of Helios~\cite{adida:2008}
did not use homomorphic tallying, but relied on mixnets to
guarantee voter anonymity. ~\cite{bulens:2011}

\section{Voting Workflow}
\label{sec:workflow}

The election workflow in Zeus consists of the following phases:
election preparation, voting, processing, and tallying.

\subsection{Preparing the Election}

\subsubsection{Initialisation}

The election administrator initialises the election in Zeus by
entering the polling date (or dates), the members of the election
committee, the name of the election and any other information required
for defining the process. The election administrator may be a member
of the election committee.

\subsubsection{Notifications to the Election Committee}

The election administrator uses Zeus to send notifications to the
election committee members. The notifications contain URLs by which
the committee members can access the system in order to generate and
upload their encryption keys.

\subsubsection{Generation of Election Keys}

Committee members generate their encryption keys. The encryption keys
are generated in their browsers and the members are instructed to save
them in a secure place. They then upload the public parts of their
keys to Zeus, to be used for encryption the voters' ballots when polls
open. 

\subsubsection{Ballot Setup}

The election administrator in coordination with the election committee
sets up the election ballots. This includes defining the number of
ballot questions and acceptable answers, whether the order of ballot
choices will be important (as in STV) or not (as in approval voting),
and the contents of each ballot.

\subsubsection{Voters Addition}

Voters are entered to the system by the administrator and the election
committee by means of a CSV file (typically generated through a
spreadsheet application) containing, for each voter the voter's email
and name.

\subsubsection{Election Freeze}

Finally the election is frozen; when this happens, the voters receive
at their mailboxes a message inviting them to cast their vote. The
message contains a link that leads them to the voting booth, where
they can compile their ballot. The voting booth opens at the date and
time appointed by the election authority; if the voters visit the link
before, they are informed accordingly.

\subsection{Voting}

Voters access the voting booth via the link they have received. They
compose their ballot, which they proceed to submit. All the process
happens at the client browser, with no interaction with the server
until the ballot has been encrypted and sent to be stored along with
the rest of the election's ballots. The user then receives an e-mail
containing a receipt for the ballot. The receipt contains the
cryptographic data that can be used to verify that the ballot has been
counted in the results.

\subsection{Tallying}

When the polls close the administrator and the election committee can
proceed to mix and decrypt the ballots. Mixing is carried out at least
once, on the Zeus server, and can optionally be carried out in
additional independent servers. Independent mixing works by means of a
command line tool to which a URL with the encrypted ballots is passed;
the tool mixes them and returns them back to Zeus. 

Once all mixes have been completed, the election committee members
receive an e-mail notification to proceed to decrypt the ballots. The
notification contains a URL link through which the encrypted ballots
are downloaded to the client browser, partially decrypted there, and
returned back to Zeus.

After the decryption Zeus provides a tally of the decrypted and
anonymised ballots, along with all the cryptographic proofs that can
be used to verify and audit and election.

\subsection{Ballot Casting and Encoding for Preference Elections}

In Helios, ballots consist of answers to binary ``yes'' or ``no''
questions, framed in the appropriate way. For instance, a voter
indicates $k$ out of $n$ candidates on a ballot by selecting them, in
which case the answer ``Would you like X as a Y?'' gets a yes (1),
otherwise a no (0). In STV, as in any system in which we would need
the whole ballot to be decrypted in the end, the ballot must be
encoded in some way. We encode each ballot as a integer by assigning a
unique number to each possible candidate selection and ranking. The
total number of possible ballots is $p _{n1} + p_{n2} + \cdots + p
_{nk}$, where $p_{nk}$ is the number of sequences of $k$ objects out
of $n$, that is $p_{nk} = n(n - 1)\cdots(n - k + 1)$\footnote{The
  number $p_{nk}$ is also written $n^{\underline{k}}$, or $(n)_k$,
  called ``Pochhammer's symbol''~\cite[p.\ 48]{graham:1994}. In closed
  form it is $p_{nk} = \sum_{k=0}^{n} (-1)^{n-k}\left\{n \atop
    k\right\}x^k$, where $\left\{n \atop k\right\}$ is the Stirling
  number of the first kind~\cite{weisstein:pochhammer}.}. As each
ballot is sent in encrypted form to the Zeus server, for its encoding
number $b$ we must have $b \in [0, 10^p]$, where $p$ is the order of
the group used in the ElGamal encryption scheme, so the encoding does
not present a practical limit in real elections (if it did, we could
always break the number in parts and send them separately).

The implementation of the encoding follows closely the mathematical
definition. We encode each ballot as an integer by enumerating the set
$\mathcal{E}$ of all possible ballots. We take $k$ choices out of $n$,
for $0\leq k \leq C$, where $C$ is the maximum choices allowed in the
ballots \footnote{$N=C$ for the elections we hosted}, and we then take
their $k!$ permutations. Summing it up, all the possible ballots are
\begin{equation}
\label{eq:max_encoded}
|\mathcal{E}| = \sum^{C}_{k=0}\binom{n}{k}k!
\end{equation}
When enumerating, we first count the smaller selections (i.e. take
\textit{zero} candidates, then \textit{one}, then \textit{two},
\ldots) so that for small numbers of choices the encoded ballot will
have a small value, thus saving valuable bit-space.\footnote{ For
  example, selecting up to all of 300 candidates needs 2048 bits,
  while selecting 10 out of 1000 candidates needs only 100 bits.}

% gtsouk: describe the exact encoding / decoding algorithm for the
% ballots. 

\subsection{Ballot Casting and Encoding for Approval Elections}

Although the prime impetus was to support preference elections, it was
later required of Zeus to support simple approval elections, where
voters simply select candidates from several party lists without
specifying an order of preference. Such elections are already
accommodated by the existing Helios implementation, and it would be
possible to fallback to it, therefore using homomorphic encryption for
approval elections and mixnets for preference elections.

We decided to keep the mixnet model for two reasons. First,
implementation would be easier and safer, as we had in the meantime
departed from the Helios implementation to a point where considerable
refactoring and testing would be required to re-use it. Secondly,
mixnets allow us to guarantee anonymity by making the election
authorities part of the anonymisation process, when they decide to
ran a mixnet themselves.

The encoding scheme for approval elections was derived by embedding it
into the existing encoding for preference voting. The party lists were
fitted each with an extra entry to denote voting for a party without
voting for any candidates. Then each candidate name was prefixed with
the party name and the party lists were concatenated to form the final
candidate list. The encoding for preference voting can represent any
ordered choices of candidates across party-lists, while approval voting
only selects from one party list with no order. To counter this
redundancy we enforced a canonical form for the plaintext.\footnote{The
user interface never produces an invalid vote, but the server always
validates this.}
The alternative was to create a non-redundant encoding for approval
voting, again via enumeration of possible ballots. However, the
introduction of new cryptographic primitives would require considerably
more time in development and testing.


\section{Internal Architecture}
\label{sec:architecture}

Zeus consists of a core workflow module, the web application,
and the web user interface.
The workflow module implements an abstract version of the complete Zeus
workflow, along with all the required cryptographic functions and
validations. The web application extends the abstract workflow module
and builds a usable voting system upon a database.
The web application handles authentication, election formulation, vote
submission, and trustee interactions for key setup and decryption.
The user interface offers access to the web application through
a web browser, and localy performs vote preparation and encryption,
as well as trustee key generation and decryption.

\subsection{Poll Stages}
The abstract workflow represents each poll as a structured document,
and defines five consecutive stages for it, from creation to completion.
At each stage more data are recorded into the document,
either coming from the outside, such as votes,
or are the results of processing existing data, such as vote mixing,
or both, such as final ballot decryption.

Once recorded in the poll document, data are never modified,
and each stage is validated upon transition into the next one.

At the final stage,
the document represents a full account of a completed poll,
containing everything needed to verify the process,
from votes, to results, to proofs.

\subsection{Creating}
At this stage, the candidates, voters, and trustees are recorded
into the document.

The candidate list is just an ordered list of unique candidate names.
The names are not important and are not interpreted. However,
different types of elections put structure within the candidate list
which is interpreted at tallying time to verify and count the ballot,
as discussed in Section~\ref{poll_types}.

The voters are likewise represented by uninterpreted but unique names.
For each voter,
Zeus generates several random numbers, the \emph{voter codes}, that can
independently be used both for audit votes and voter authentication,
as discussed in Section~\ref{audit_codes}.

For trustees, only their public key is recorded.
For each poll, and additionally to any external trustees,
Zeus always creates and registers a trustee key for the poll.
As per the Helios workflow, this allows the service operator to
provide anonymity guarantees in addition to those provided by
the appointed committee of trustees.

\subsection{Voting}
\label{voting}
The poll enters the \emph{voting} stage once all the trustee public
keys have been combined\footnote{by multiplication} to create the
\emph{poll public key}.

After that, the trustees, voters, and candidates cannot be modified.

Once the poll public key is available,
voters can encrypt and submit their votes,
as discussed in Section~\ref{vote_preparation}.

For each submitted vote, zeus generates a cryptographically signed
receipt for the voter.
With the receipt, the voter can later prove that his vote submission was
acknowledged as successful, perhaps in the context of an investigation
for a complaint.
Receipts are detailed in Section~\ref{receipts}.

Each voter may submit a vote multiple times.
Each time, the new vote replaces the old one,
and the new receipt explicitly states which vote it replaces.
No vote can be replaced more than once.

Apart from normal votes, voters may submit \emph{audit votes} to verify
that their local system really does submit the vote they choose,
and that it does not alter it in any way.
The process is detailed in Section~\ref{audit_codes}.

\subsection{Mixing}
\label{mixing}
After voting has ended, the poll is put into the \emph{mixing} stage,
where the encrypted ballots are anonymized.
Only the ballots eligible for counting are selected, by excluding
all audit votes, all replaced votes, and all votes by excluded voters.

Voter exclusion is a digital convenience due to
the voter's identity still being known after voting has ended.
Anonymity is achieved at the end of the mixing stage,
and not at cast time as in traditional ballot box polls.

The poll's officials may choose to disqualify a voter if
it is discovered that he was wrongly allowed to vote, or if
his conduct during the election was found in breach of rules.
Neither the voter nor their votes are deleted from the poll document.
Instead, the voter is added into an exclusion list along with
a statement justifying the decision to exclude them.

The eligible encrypted votes are first mixed by Zeus itself
by a Sako-Kilian mixnet~\cite{sako:1995}.
The bulk of the computational work in mixing is to produce
enough rounds of the probabilistic proof.
The work is distributed to a group of parallel workers,
one round at a time.
\footnote{Our deployment used 128 rounds and 16 workers.}
Verification of mixes is likewise parallel.

After the first mix is complete, additional mixes by external agent
may be verified and added in a strict order in the mix list.
Zeus includes a command line utility that can perform external mixes.

\subsection{Decrypting}
\label{decrypting}

After mixing is complete, the final mixed ballots are exported to
the trustees to be partialy decrypted.
The resulting decryption factors are then verified and imported
into the poll document.
Zeus' own decryption factors are computed last.

\subsection{Finished}
\label{finished}
In the transition to \emph{finished}, all decryption factors are
distributed in parallel workers to be verified and combined together
to yield the final decrypted ballots.
The decrypted ballots are recorded into the poll document,
and then the document is cast into a canonical representation in text.
This textual representation is hashed to obtain a cryptographically
secure identifier for it, which can be published and recorded
in the proceedings.


\subsection{Software Dependencies}
Random number generation \emph{(Fortuna)}, primality testing
\emph{(Miller-Rabin)} and modulo inversion from \texttt{PyCrypto}.
Exponentiation from \texttt{libgmp} via \texttt{gmpy} (really faster
than Python's builtin \texttt{pow}). We have implemented all other
operations in native Python and the Python standard library, including
hashing. We have studied \texttt{Helios}, \texttt{PloneVoteCryptoLib},
\texttt{PyCrypto} for our implementation.

% TODO: References to softwares
% signed canonical representation of proof data
% enumerated encoding: only valid votes are encoded (less than limit)
% redundant encoding: decoded vote is checked for canonical form.
% browser compatibility.

% \subsection{Cryptosystem}
% Zeus uses the ElGamal cryptosystem on the prime-order-$q$ subgroup
% $\mathcal{G}$ of the quadratic residues of a safe prime $p = 2q + 1$,
% such that
% $$m \in \mathcal{G} \longleftrightarrow \mathcal{L}(m) = m^q = 1 \mod p$$
% $\mathcal{L}$ being the \emph{Legendre} symbol.
% For reference, we reproduce all the primitives we use in a table.
% All base numbers are in $\mathcal{G}$, all exponents in $\mathbb{Z}^{*}_q$,
% and all operations are $\mod p$, unless explicitly noted.
% We group $x=a, y=b, \ldots$ as $(x,y,\ldots)\equiv(a,b,\ldots)$.

% \begin{tabular}{rl}
% \textbf{modulus}       \, & $p \equiv 3(\mod 4)$ \hfill\textit{\small(safe prime)}\\
% \textbf{generator}     \, & $g: g^q = 1$      \\
% \textbf{order}         \, & $q = (q-1)/2$\hfill\textit{\small(ElGamal group prime order)}\\
% \textbf{secret}        \, & $x$               \\
% \textbf{public}        \, & $y = g^x$         \\
% \textbf{committment}   \, & $t$               \\
% \textbf{challenge}     \, & $c$               \\
% \textbf{response}      \, & $f$     \hfill\textit{\small(from proof)} \\
% \textbf{\parbox{7em}
%     {\setlength{\baselineskip}{.85\baselineskip}
%      \raggedleft
%      group \\
%      encoding}}        \, & $T: x \mapsto \left\{
%                             \begin{matrix}  x &,&\quad x^q=1 \\
%                                            -x &,&\quad x^q\neq 1
%                             \end{matrix}\right.\quad
%                             T^{-1}: e \mapsto \left\{
%                             \begin{matrix}  e &,&\quad e \leq q \\
%                                            -e &,&\quad e > q \\
%                             \end{matrix}\right.
%                             $ \\
% \textbf{secret nonce}  \, & $r, w$  \hfill\textit{\small(in encryptions and proofs)} \\
% \textbf{ciphertext}    \, & $(a, b)\equiv(g^r, y^r m$)
%                                     \hfill\textit{\small(encryption)} \\
% \textbf{plaintext}     \, & $m = a^{-x}b$
%                                     \hfill\textit{\small(decryption)} \\
% \textbf{reencryption}  \, & $(a',b')\equiv(g^{r'}a, y^{r'}b)$ \\
% \textbf{hashing}       \, & $\mathcal{H}(n_0n_1n_2\ldots)$
%                             \hfill\textit{\small(hash textual representation of numbers)} \\
% \textbf{discrete log}  \, & $y = g^x \Rightarrow \log_gy=x \;\leadsto\;$
%                                     \textbf{\small prove you know} $x$ \\
%                        \, & $(t, c, f) \equiv (g^w, \mathcal{H}(t), w+xc)$
%                                     \hfill\textit{\small(prove knowledge)} \\
%                        \, & $g^f \overset{?}{=} ty^c$
%                                     \hfill\textit{\small(verify knowledge)} \\
%                        \, & $g, y, u, v \;\leadsto\;$
%                                     \textbf{\small prove} $\:log_gu = log_yv = w$
%                                     i.e. $(g, y, g^w, y^w)$ \\
%                        \, & $(t_g,t_y,c,f) \equiv (g^w, y^w, \mathcal{H}(t_gt_y), w+xc)$
%                                     \hfill\textit{\small(prove equality)} \\
%                        \, & $g^f \overset{?}{=} t_gu^c \wedge
%                              y^f \overset{?}{=} t_yv^c$
%                                     \hfill\textit{\small(verify equality)} \\
% \textbf{signature}     \, & $(z, s)\equiv\big(g^{w}, w^{-1}(m-zx)\mod (p-1)\big)
%                                     \quad w = 2w'-1,\: 3\leq w'\leq q$ \\
%                        \, & $m^s \overset{?}{=} y^zz^s$
%                                     \hfill\textit{\small(verify signature)} \\
% % TODO: cite Fiat-Shamir, Schnorr, Chaum-Pedersen, HAC in this table
% \end{tabular}

% \subsection{Creating}
% \subsection{Voting}
% \subsection{Mixing}
% \subsection{Decrypting}
% \subsection{Validation}

\subsection{Election User Interface}

The users of Zeus were not expected to be experts in cryptography, or
to have any knowledge of computer science; we could only assume
familiarity with internet browsing, since the electoral body comprises
people with heterogeneous characteristics. A significant design goal
was therefore \emph{interface}, and \emph{workflow} simplicity. At the
same time, the knowledgeable voter needed access to all information
and functions needed to both understand and verify the process.

\subsection{Zeus Users}

Zeus distinguishes the following classes of users:

\begin{itemize}
\item Election administrators
\item Election committee members
\item Voters
\end{itemize}

Election administrators are responsible for setting up an election;
that is, starting a new election instance in Zeus, setting the date
for the election, entering the particulars of the election committee
members and the list of voters. During the election the administrators
are responsible for updating the list of voters (i.e., adding,
removing, or updating the details for a voter) and extending the poll
times, if necessary. When polls close the administrators are
responsible for starting the ballot mixing and vote count process.
Election committee members hold the cryptographic keys for the
election. Before the election starts they have to create their keys
and upload the public parts to the Zeus server. During decryption each
election committee member partially decrypts the ballots using the
private part of the key. Voters visit the voting booth to cast their
vote; they can vote repeatedly until polls close.

Election administrators access Zeus through a username and password.
Election committee members and voters access Zeus through URL links
that are sent to them by the system, as we see below.

% \subsection{Administrator View}
% \subsection{Voting Booth}
% \subsection{Auditing}

% \section{Ballot Submission}

% The submitted ballot contains
% %gtsouk: describe the conntents of the submitted ballot
% A submitted ballot is a JSON object of the form:
% \begin{verbatim}
% {
%   answers : [];
%   election_hash : ;
%   election_uuid : ;
% }
% \end{verbatim}

% \section{Election Verification}

% %gtsouk: describe the election verification algorithm

% \section{Implementation Considerations}

\section{Experience}
\label{sec:experience}

From the voter's point of view, there has been a published usability
analysis of Helios~\cite{karayumak:2011}. From our part we did not
receive any particular complains from the voters, except for one: that
the system would not work with Internet Explorer. The users were
informed of that in the notifications they received, but apparently
not all of them noticed it, or knew exactly how to download and
install one of the supported browsers (recent editions of Firefox,
Chrome).

Zeus was used in 23 elections that took place in 22 institutions
around Greece (see Table~\ref{table:zeus-elections}). Elections were
carried out successfully in all planned institutions. This was a
significant success, as the stakes were particularly high and the
voting process charged with emotions. To understand that it is
necessary that we provide some surrounding context.

The law that instituted the Governing Councils in Greek educational
institutions met resistance from some political parties, student and
academics unions. A widespread means of protest was to disrupt the
election of Governing Councils, so that they could not be inaugurated.
Successive aborted elections pushed forward the adoption of electronic
voting, which in turn was the subject of controversy. 

The controversy meant that some political parties were against the use
of Zeus, or any form of electronic voting. As a result, Zeus was the
subject of a number of public denouncements and letters, detailing
weaknesses of electronic voting systems in general and Helios in
particular. These did not challenge Zeus, as the weaknesses applied to
older versions of Helios. Altough Zeus was open sourced from the start
and we welcomed constructive criticism or the exposure of unknown
vulnerabilities, we did not receive any. The atmosphere was at times
vitriolic, as when a mainstream political party called the people
behind Zeus ``techno-fascists'', thus introducing a new term in the
greek lexicon and bemusing the people involved in the whole effort.

This led to attacks on Zeus, and the voting process supported by Zeus.
None of them was eventually successful. We describe them briefly
in~\ref{ssec:attacks}. A discussion on other security-related aspects
follows in~\ref{ssec:security-discussion}. We round up with a
presentation of voting logistics~\ref{ssec:logistics}.

\subsection{Attacks}
\label{ssec:attacks}

\subsubsection{University of Thrace}

The elections at the University of Thrace had been planned for October
24, 2012. Due to a coding bug, the polls opened the at the time the
election was frozen, on October 22. This was discovered by some
voters, who went on to vote before the official poll opening. The
issue was publishised and the election was annulled and repeated at a
later date. Technically, the election could have proceeded without a
problem, since there was no way anybody could have tampered with the
election results, but the issue was sensitive and politicised, so that
the election committee took the most cautious route.

Elections were called again for October 29, 2012, with polls opening
at 9:00. At dawn our Networks Operations Center noticed a large number
of connections to Zeus. Upon investigation, it was found that Zeus was
the subject of a slowloris attack~\cite{slowloris}. The attach was
swiftly dealt with, and apart from some initial inconvenience did not
cause any real problems. Moreover, the attackers used static IP
addresses and could easily be traced and blocked.

\subsubsection{University of the Aegean}

The University of the Aegean had planned its elections for November 2,
2012. After freezing the election, and when the notifications to the
voters had been sent, voters started receiving additional
notifications containing bogus URLs, from a cracked university server.
Although the bogus URLs were not functional, a number of users were
confused. The electoral committee decided to cancel the elections and
call new elections at a later date. To avoid similar situations in
coming elections we recommended the use of Sender Policy Framework
(SPF)~\cite{rfc4408} to institutions that had not been using it.

The repeat elections were held on November 9, 2012. The authorities at
the university had authorised the set up of a new mail server, to be
used solely for the elections, in which all voter's were opened
accounts. The mail server was hosted outside the university's
premises. This avoided a problem that would have risen from a sit-in
at the building housing the main mail server, which had been switched
off by protesters at election day.

\subsubsection{Agricultural University of Athens}

On election day, November 5, 2012, protesters staged a sit-in at the
mail server building of the Agricultural University of Athens. The
mail server was switched off, cutting off access to e-mail, and
depriving voters that had not downloaded the access link from voting.
The university authorities responded by asking voters to come
physically to a location outside the campus, where they could be
issued with a new URL.

\subsubsection{University of Patras}

The University of Patras held its elections on November 7, 2012. Just
prior to the opening of the polls a slowloris attack as noticed and
dealt with.

\subsubsection{University of Athens}

A little while after polls opened at the University of Athens, on
November 12, 2012, protesters staged a sit-in at the university's
Network Operations Server and cut-off internet access. This took
offline mail access to members of the university and, like in the
Agricultural University of Athens, deprived voters that had not
downloaded the access link from voting. More dramatically, the
internet shutdown affected all sorts of university services, including
internet connectivity to university hospitals.

The university responded by extending polling by three days. At the
same time, an alternative voter's notification scheme was implemented,
via which the voters would receive instructions for voting via SMS.
The sit-in was lifted on the last election day, as it was realised
that the elections would go on regardless.

\subsubsection{National Technical University of Athens}

A sit-in similar to the one at the University of Athens was planned at
the National Technical University of Athens. However, it did not last
long, as the SMS-based infrastructure was already in place, so that it
would be irrelevant, and the university authorities made it clear that
they would not tolerate the disruption.

A different, more insidious kind of attack was revealed later on. A
few days before the election, one voter had contacted us complaining
that the link he had received did not seem to be functioning at all
(it should link to a page notifying the user about the coming
elections; instead, the user was redirected to the main Zeus page).
Two days before the election it was discovered that the user's e-mail
stored in Zeus had been changed, and a new e-mail had been sent to the
newly entered mail address. The user complained that he never used
that adddress, so another URL was issued and sent to his institutional
mail address. The voter voted without a problem on election day.

A few days afterwards a protesters' site announced they had
circumvented Zeus security by taking hold of a voter's URL after
contacting the election committee and asking them to update the
voter's contact details. That explained the e-mail change, and it
meant that it was the result of a social engineering attack. Luckily,
it had been caught in time---something the protesters did not let on.

\begin{table*}[t]
  \begin{threeparttable}
    \begin{tabular}{llllll}
      \hline
      Institution & Voters & Voted & Ballots & Open & Close\\
      \hline
      Harokopio University & 61 & 59 & 63 & 2012-10-19 10:00 & 2012-10-19 11:00\\
      TEI Piraeus & 144 & 102 & 111 & 2012-10-22 09:00 & 2012-10-22 13:00\\
      Ionian University & 105 & 92 & 101 & 2012-10-25 09:00 & 2012-10-25 15:00\\
      TEI Patras & 93 & 78 & 79 & 2012-10-26 09:00 & 2012-10-26 13:00\\
      University of Thrace & 583 & 490 & 528 & 2012-10-29 09:00 & 
      2012-10-29 16:00\\
      TEI Athens & 487 & 437 & 470 & 2012-10-31 08:00 & 2012-10-31 18:00\\
      University of Thessaly & 429 & 328 & 345 & 2012-10-31 09:00 & 
      2012-10-31 14:36\\
      Panteion University & 239 & 181 & 183 & 2012-10-31 09:00 & 
      2012-10-31 16:00\\
      University of Thessaloniki & 2065 & 1580 & 1632 & 2012-11-01 09:00 
      & 2012-11-01 16:00\\
      Athens University of Economics and Business & 195 & 172 & 181 & 
      2012-11-02 10:00 & 2012-11-02 17:00\\
      Agricultural University of Athens & 182 & 102 & 110 
      & 2012-11-05 09:00 & 2012-11-05 16:00\\
      University of Ioanina & 536 & 426 & 443 & 2012-11-05 09:00 &
      2012-11-05 17:00\\
      University of Crete & 493 & 361 & 373 & 2012-11-05 09:00 & 
      2012-11-05 18:00\\
      University of Macedonia & 169 & 163 & 172 & 2012-11-06 10:00 & 
      2012-11-06 18:00\\
      University of Patras & 703 & 532 & 555 & 2012-11-07 09:00 & 
      2012-11-07 17:00\\
      Athens School of Fine Arts & 46 & 40 & 41 & 2012-11-08 07:00 & 
      2012-11-08 16:00\\
      University of the Aegean & 292 & 161 & 172 & 2012-11-09 09:00 & 
      2012-11-09 17:00\\
      University of Athens & 1897 & 1504 & 1618 & 2012-11-12 09:00 & 
      2012-11-15 18:00\\
      University of Piraeus & 183 & 180 & 199 & 2012-11-14 09:00 & 
      2012-11-14 15:00\\
      University of the Peloponnese & 128 & 124 & 128 &
      2012-11-22 07:00 & 2012-11-22 17:00\\
      Technical University of Crete & 126 & 116 & 127 & 
      2012-12-04 10:00 & 2012-12-04 15:00\\
      National Technical University of Athens & 536 & 351 & 363 &
      2012-12-10 09:00 & 2012-12-10 17:00\\
      TEI Piraeus & 143 & 104 & 105 & 2012-12-21 09:00 & 2012-12-21 17:00\\
    \end{tabular}
    \begin{tablenotes}
      \small 
      \item The number of ballots may be larger than the number of
      people voted when voters have voted multiple times.
      \item TEI is Technological Educational Institute. 
    \end{tablenotes}
  \end{threeparttable}
  \caption{Zeus Held Elections\label{table:zeus-elections}}
\end{table*}

\subsubsection{Recreate Greece}

Encouraged by the success of Zeus in previous elections, a new
political party, ``Recreate Greece'' (``dimiourgia, xana!'' in Greek)
approached us and enquired whether they could use Zeus for the
elections leading to, and during, their party congress. Although not
among our target constituency, we decided to allow the use of Zeus in
a ``best-effort'' basis from our part, since that entailed the
development of new, interesting functionality---specifically, party
list elections, elections with multiple questions and answers, and
elections where blank votes are not allowed.

It turned out that in one of their elections a voter wanted to show
that e-voting cannot be trusted since anybody can vote with a voter's
access credentials, if these are leaked. On April 6, 2013, the voter
posted his voter URL on a Facebook page. The electoral committee noted
the incident, and therefore moved to disqualify the ballots cast under
that voter's name. This functionality is offered by Zeus as the attack
vector had already been anticipated; it is possible to disqualify a
vote before mixing.

\subsection{General Security Issues}
\label{ssec:security-discussion}

As explained above, Zeus was born in controversy, and its use was
accompanied by much naysaying. An important argument against the
system was that we were not to be trusted. After all, to the voters,
as well as to each election committee, Zeus is a black box running in
some remote virtual machine. It could be that we were really running a
show: instead of Zeus voters were interacting with something else, and
we were putting on a pretence of anonymity and security. 

As we were not aware of a practical method to verify the integrity of
a running virtual machine, doubters were presented with two choices:
either they would have to take us on our word that we were indeed
honest, or they could download the code and run it themselves. Nobody
took the second option. 

Still, at several times throughout different elections we became aware
that people did not really believe us in our assertions that we were
honest and that the system behaved as we had argued it would. In two
distinct occasions members of the election committee discovered that
their election key could not be read by Zeus---a bad file. We had
instructed them to keep two copies on two different {\sc usb} sticks.
Luckily, in both instances the back-ups were readable. People, though,
were to various degrees sure that somehow we would be able to ``do
something'' about it, disregarding our assurances that the only thing
we could really do would be to re-run the elections from the beginning.

A similar veil of doubt appeared when people called us condidentially
during elections, and shortly after the ballot box had been closed, to
ask us ``how things were going'', and whether we had an indication of
the results. We were at pains to explain that there is no way we would
have such knowledge.

Zeus gives to the election committee a full set of cryptographic
proofs that can be used to verify the results. We had stressed that
these proofs would be the sole arbitrer of any dispute, and the only
material that could be really trusted. At some point in the elections,
though, we decided to provide, just for convenience, a {\sc pdf}
summary output of the results. We found out to our chagrin that a
plain, unsigned {\sc pdf} document carries more weight than any of our
pronouncements.

In particular, we used ReportLab's open source libraries for {\sc pdf}
creation. In one election (elections for president of the Piraeus
Technical Educational Institute, held on March 26, 2013), a
disgruntled candidate complained to the election committee that the
{\sc pdf} results' creation date was before ballot closing time. It
turned out that he was right, and that this was due to a bug with the
{\sc pdf} libraries (which we reported to ReportLab). The fact that
we never meant the {\sc pdf} document to be the definitive results
record of an election seemed to matter little. The issue was resolved
after we issued an official explanation, complete with details of the
bug and possible fixes. The authority of the {\sc pdf} document is
something we came up against repeatedly: people did not seem to care
for any other kind of authority apart from that given by the printed
matter.

\subsection{Logistics}
\label{ssec:logistics}

Zeus is offered as a hosted service by {\sc grnet}; hosting here
entails more than just a well-endowed virtual machine. Elections are a
sensitive matter, since people exercice their basic democratic right.
They must feel secure, and they must feel they are not slighted by
their lack of technical knowledge.

In every Zeus election we had a helpdesk in place before the election
started, during the election itself, and after the ballot box closed.

Before the election started the helpdesk was tasked with helping the
election committee setting up the election. Although we have produced
manuals, for both the election committee and voters, people frequently
requested more engagement from our part. As an aside, we should point
out that {\sc rtfm} rules, as always, and the more so the more
technical a user is. We found that users with no particular computing
experience were more careful in their use of the system and more
consientious in their use of the manuals provided, than users who,
emboldened by their prowess in computing went on without bothering to
consult the documentation.

Our helpdesk colleagues helped the election committee to generate
their keys and to set up the voters list, ensuring that the input
file, a simple {\sc csv} document, was well formed---not a trivial
matter, as it turned out, as despite our detailed instructions college
registrars often failed to produce an up to date list of eligible
voters with their current e-mails. Once the e-mails containing the
voters links had been sent the helpdesk handles bounces. The bounces
are forwarded to the election committee with the voter links removed,
so that the election committee can contact the voters and correct
their e-mail addresses but cannot abuse the system and start voting in
their stead. 

It could be possible, in this setting, for the administrators of Zeus
not to forward the bounces and to therefore compromise the system by
doing the voting themselves. This, though, would require of us to know
in advance which voters would not bother to contact the election
committee enquiring why they have not received their voter links, at
which point the election committee would discover that somebody else
has been voting for them. Even if we were crooks we would not run a
risk as high as this one.

During the election the helpdesk guided the election committee when
they were in doubt about how to correct a voter's details. This
happened when a voter contacted the election committee asking them to
change their e-mail address to another one. 

Then, at the end of each election process, the helpdesk was sometimes
asked to show the steps for the mixing and decryption process, as well
as the presentation of the actual results.

In general, even though the helpdesk workload was not particularly
high, it had special time requirements, such as working out of normal
business hours to ensure that the communication with the election
committee and the voters was as prompt as possible. Both election
committees and voters can get very anxious when things are not working
in the way they may have anticipated.

% gtsouk: Including timing measurements, especially wrt differen versions of
% Python, GMP, etc.

% gtsouk: also instances where existing implementations were not up to
% par from a cryptographic point of view

\section{Discussion}

\section{Availability}

Zeus is open software, available at \url{https://github.com/grnet/zeus}.

{\footnotesize
\bibliographystyle{acm}
\bibliography{zeus}
}


\end{document}

